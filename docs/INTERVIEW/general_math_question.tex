\documentclass[a4paper]{article}
\usepackage[margin=25mm]{geometry}
\usepackage[none]{hyphenat}

\usepackage{amsmath}
\usepackage{amsfonts}
\usepackage{amssymb}
\usepackage{graphicx}
\pagenumbering{gobble}
\usepackage{verbatim}

\usepackage{sectsty}
\sectionfont{\fontsize{11}{12}\selectfont}

\title{GENERAL INTERVIEW QUESTIONS TEMPLATE}
\author{KHALID SHAFIQ  \\
        \small Quant Analyst @Barclays Capital \\
        \small QA Securitized Products - Agency Modeling \\
}
\date{} % Comment this line to show today's date

\begin{document}
\maketitle

\begin{abstract}
This document provides a short summary of questions categorized based on topics from different area of mathematics as well as asset classes. Different topics and level of difficulty can be set by interviewer based on candidate level of experience and exposure to different asset classes. 
\end{abstract} \hspace{10pt}
     
\section*{CALCULUS}
\begin{enumerate}
\item Let $f(h) = \sqrt{1+h}$ and $g(h) = sin(h)$. Show that first order Taylor approximation of
$f(h).g(h) = h + h^2/2 + \mathcal{O}(x^3)$ as $h \to 0$.
\end{enumerate}

 
\section*{PROBABILITY THEORY}
\begin{enumerate}
\end{enumerate}

\section*{FINANCIAL ENGINEERING}
\begin{enumerate}
\end{enumerate}


\section*{COUNTER-PARTY RISK (XVA)}
\begin{enumerate}
\end{enumerate}

\section*{INTEREST RATE MODELING}
\begin{enumerate}
\end{enumerate}

\section*{PREPAYMENT MODELING}
\begin{enumerate}
\end{enumerate}

\section*{FIXED INCOME}
\begin{enumerate}
\end{enumerate}


\section*{TIME SERIES ANALYSIS}
\begin{enumerate}
\end{enumerate}

\section*{TRADING STRATEGIES}
\begin{enumerate}
\end{enumerate}


\section*{ALGORITHM ANALYSIS}
\begin{enumerate}
\end{enumerate}

\section*{C++}
\begin{enumerate}
\end{enumerate}

\section*{PYTHON}
\begin{enumerate}
\end{enumerate}

\section*{NUMERICAL ANALYSIS}
\begin{enumerate}
\end{enumerate}

\end{document}
